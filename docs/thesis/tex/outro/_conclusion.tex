% conclusion
\cleardoublepage
\phantomsection
\addcontentsline{toc}
    {chapter}
    {Conclusion}

\chapter*{Conclusion}
In summarised conclusion of this dissertation, I will now break down what work has been completed to achieve my research goal. I will outline the steps taken through each chapter, including the hurdles and solutions found to the problem or implementation.

In my first chapter, the Literature Review, I discussed the concepts behind parallelism and how it applies to my research goal. MPI was posited as a possible method of achieving my goal. I next discussed in detail the different networking technologies available to cluster developers, recommending TCP for a Beowulf system. I then presented an example implementation of hello world using Open MPI and argued why it would be most suitable for my system. I finally discussed a comparable library for creating a Beowulf cluster and justified my reasoning for using MPI before concluding chapter one.

My second chapter, the Requirements and Analysis discussion, I started by proposing my options with regards to compute hardware, finalising on a dedicated server with virtual nodes due to cost and time constraints. Next, I discussed software I would need, incorporating the existing hypervisor and associated software. I discussed what would be required of my operating system and determined a rolling distribution of Linux would best serve my needs. Finally, I discussed the custom software I would write and any requirements it would have to fulfil, as well as the networking tools and auxiliary components I would require.

The third chapter covered my system design, including both hardware configurations and software setups. To start, I proposed a network topology with five virtual compute nodes, based around existing systems in my network. An initial design of my virtual nodes was presented with justifications for its design and a high-level method of communication was established. I then conducted a small literature review and designed a method of benchmarking my system using a brute-force on the eggholder function with justification for its use over hashing.

The fourth chapter introduced my implementation and testing procedures. First, I configured the virtual network switches for use with my virtual compute nodes and prepared the virtual nodes for operating system use, choosing and justifying the most optimal settings for operations. I then proceeded with installation of Manjaro, using the Architect software to configure a minimal installation. I configured the operating system for use and installed the required packages, being careful not to incur unnecessary bloat while I complete this task. I then began to program my custom benchmarking software, first using the hello world example code to test the system's functionality followed by the chronicling and testing of major program milestones, correcting issues as I went.

In the final chapter, I expanded on chapter four by collecting and discussing results from my benchmarking software. Using time taken for completion as my metric of performance, I discussed issues in effort creep and concluded it did not threaten to majorly undermine my research before visually presenting my results, offering a prediction on the state of my cluster were I to add more nodes. The chapter is concluded by discussing my views on how the project went and any future work I might see in future based on this dissertation's findings.
